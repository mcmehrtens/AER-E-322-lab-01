% AER E 361 Mission Report Template
% Spring 2023
% Template created by Yiqi Liang and Professor Matthew Nelson

% Document Configuration DO NOT CHANGE
\documentclass[12 pt]{report}
% --------------------LaTeX Packages---------------------------------
% The following are packages that are used in this report.
% DO NOT CHANGE ANY OF THE FOLLOWING OR YOUR REPORT WILL NOT COMPILE
% -------------------------------------------------------------------

\usepackage{hyperref}
\usepackage{parskip}
\usepackage{titlesec}
\usepackage{titling}
\usepackage{graphicx}
\usepackage{graphviz}
\usepackage[T1]{fontenc}
\usepackage{titlesec, blindtext, color} %for LessIsMore style
\usepackage{tcolorbox} %for references box
\usepackage[hmargin=1in,vmargin=1in]{geometry} % use 1 inch margins
\usepackage{float}
\usepackage{tikz}
\usepackage{svg} % Allows for SVG Vector graphics
\usepackage{textcomp, gensymb} %for degree symbol
\hypersetup{
	colorlinks=true,
	linkcolor=blue,
	urlcolor=cyan,
}
\usepackage{biblatex}
\addbibresource{lab-report-bib.bib}
\usepackage{amsmath}
\usepackage{listings}
\usepackage{multicol}
\usepackage{array}

\usepackage{hologo} %KYR: for \BibTeX
%\usepackage{algpseudocode}
%\usepackage{algorithm}
% This configures items for code listings in the document
\usepackage{xcolor}

\usepackage{fancyhdr} % Headers/Footers
\usepackage{siunitx} % SI units

\definecolor{commentsColor}{rgb}{0.497495, 0.497587, 0.497464}
\definecolor{keywordsColor}{rgb}{0.000000, 0.000000, 0.635294}
\definecolor{stringColor}{rgb}{0.558215, 0.000000, 0.135316}
\definecolor{mygreen}{rgb}{0,0.6,0}
\definecolor{mygray}{rgb}{0.5,0.5,0.5}
\definecolor{mymauve}{rgb}{0.58,0,0.82}

\lstdefinestyle{customc}{
  belowcaptionskip=1\baselineskip,
  breaklines=true,
  frame=L,
  xleftmargin=\parindent,
  language=C,
  showstringspaces=false,
  basicstyle=\footnotesize\ttfamily,
  keywordstyle=\bfseries\color{green!40!black},
  commentstyle=\itshape\color{purple!40!black},
  identifierstyle=\color{blue},
  stringstyle=\color{orange},
 }

 \lstset{ %
  backgroundcolor=\color{white},   % choose the background color; you must add \usepackage{color} or \usepackage{xcolor}
  basicstyle=\footnotesize,        % the size of the fonts that are used for the code
  breakatwhitespace=false,         % sets if automatic breaks should only happen at whitespace
  breaklines=true,                 % sets automatic line breaking
  captionpos=b,                    % sets the caption-position to bottom
  commentstyle=\color{commentsColor}\textit,    % comment style
  deletekeywords={...},            % if you want to delete keywords from the given language
  escapeinside={\%*}{*)},          % if you want to add LaTeX within your code
  extendedchars=true,              % lets you use non-ASCII characters; for 8-bits encodings only, does not work with UTF-8
  frame=tb,	                   	   % adds a frame around the code
  keepspaces=true,                 % keeps spaces in text, useful for keeping indentation of code (possibly needs columns=flexible)
  keywordstyle=\color{keywordsColor}\bfseries,       % keyword style
  language=Python,                 % the language of the code (can be overrided per snippet)
  otherkeywords={*,...},           % if you want to add more keywords to the set
  numbers=left,                    % where to put the line-numbers; possible values are (none, left, right)
  numbersep=8pt,                   % how far the line-numbers are from the code
  numberstyle=\tiny\color{commentsColor}, % the style that is used for the line-numbers
  rulecolor=\color{black},         % if not set, the frame-color may be changed on line-breaks within not-black text (e.g. comments (green here))
  showspaces=false,                % show spaces everywhere adding particular underscores; it overrides 'showstringspaces'
  showstringspaces=false,          % underline spaces within strings only
  showtabs=false,                  % show tabs within strings adding particular underscores
  stepnumber=1,                    % the step between two line-numbers. If it's 1, each line will be numbered
  stringstyle=\color{stringColor}, % string literal style
  tabsize=2,	                   % sets default tabsize to 2 spaces
  title=\lstname,                  % show the filename of files included with \lstinputlisting; also try caption instead of title
  columns=fixed                    % Using fixed column width (for e.g. nice alignment)
}

\lstdefinestyle{customasm}{
  belowcaptionskip=1\baselineskip,
  frame=L,
  xleftmargin=\parindent,
  language=[x86masm]Assembler,
  basicstyle=\footnotesize\ttfamily,
  commentstyle=\itshape\color{purple!40!black},
}

\lstset{escapechar=@,style=customc}


\titlelabel{\thetitle.\quad}

% From here on out you can start editing your document
\newcommand{\subtitle}[1]{%
  \posttitle{%
    \par\end{center}
    \begin{center}\LARGE#1\end{center}
    \vskip0.5em}%
}

\title{\textbf{Iowa State University
\\{\Large Aerospace Engineering}}}
\subtitle{AER E 322 Lab 01\\
		  Practice Experiment and Data Analysis}
\author{Matthew Mehrtens, Peter Mikolitis, and Natsuki Oda}

% Define the headers and footers
\setlength{\headheight}{70.63135pt}
\geometry{head=70.63135pt, includehead=true, includefoot=true}
\fancypagestyle{plain}{
	\fancyhead{}\fancyfoot{} % clears the headers/footers
	\fancyhead[L]{\textbf{AER E 322}}
	\fancyhead[C]{\textbf{Aerospace Structures Laboratory Report}\\
					 \textbf{Lab 01 Practice Experiment and Data Analysis}\\
					 Section 4 Group 2\\
					 Matthew Mehrtens, Peter Mikolitis, and Natsuki Oda\\
					 \today}
	\fancyhead[R]{\textbf{Spring 2023}}
	\fancyfoot[C]{\thepage}
}
\pagestyle{fancy}
\fancyhead{}\fancyfoot{} % clears the headers/footers
\fancyhead[L]{\textbf{AER E 322}}
\fancyhead[C]{\textbf{Aerospace Structures Laboratory Report}\\
			  \textbf{Lab 01 Practice Experiment and Data Analysis}\\
			  Section 4 Group 2\\
			  Matthew Mehrtens, Peter Mikolitis, and Natsuki Oda\\
			  \today}
\fancyhead[R]{\textbf{Spring 2023}}
\fancyfoot[C]{\thepage}

\begin{document}
\maketitle
\tableofcontents

\chapter{Pre-Lab} \label{pre-lab}
\section{Introduction} \label{introduction}
Aircraft wings undergo oscillations and other random forces while in flight. This lab replicates and analyzes some of the forces and oscillations a wing will experience in flight while also serving as an introduction to the PASCO tool kits and data processing. To simulate the wing, we used a cantilevered aluminum beam, and to generate and measure the oscillations, we used a PASCO tool kit---specifically the PASCO wave driver, displacement sensor and motion sensor. There were three rounds of testing; each additional round of testing introduced a new variable into the beam movement that changed the shape of the data. The data was collected using the PASCO tool kit and software provided. After the lab, we analyzed and processed the data in Python to how each variable effected the oscillation of the beam.

\section{Objectives} \label{objectives}
During this lab, our primary objectives were to:
\begin{enumerate}
	\item Learn how to record data under dynamic conditions and analyze or post-process the data.
	\item Observe approximately how a common aerospace structural material might respond to oscillatory forces.
	\item Gain familiarity with the PASCO tool kit and the PASCO Capstone software.
\end{enumerate}

\section{Hypothesis} \label{hypothesis}
\subsection{Test 1} \label{test_1}
We predict this test will provide the cleanest data of the three tests. Since the only force acting on the beam should be from the wave driver, we expect the displacement graph to shown a uniform and steady wave---matching the oscillations of the wave driver. The data from this test should closely match the oscillations of an airplane wing in very steady flight.

\subsection{Test 2} \label{test_2}
This test adds a spring-loaded weight to the cantilevered beam. Due to the oscillations of the spring loaded weight, we expect to see sudden highs and lows in the data corresponding with when the spring-loaded weight is in compression or tension respectively. The data from this test should demonstrate the oscillations of the wing in steady flight if there is an additional oscillatory or vibrational force simultaneously acting on the wing.

\subsection{Test 3} \label{test_3}
This test is similar to Test 2 (see section \ref{test_2}) except a third significant force has been introduced. Due to the addition of arbitrary impulses being applied by hand to the free end of the beam, we expect the data to show large peaks and dips in the data correlated with the timing of the impulses. The data from this test should demonstrate the oscillations of real flight as described in section \ref{test_2} but also how the wing might react during periods of high turbulence where sudden, large impulses may act on the wing.

\chapter{Lab Work} \label{lab_work}
\section{Variables} \label{variables}
\subsection{Independent Variables} \label{independent_variables}
\subsubsection{Wave Driver Frequency} \label{wave_drive_frequency}
The frequency of the wave generated by the PASCO wave driver.

\subsubsection{Wave Driver Amplitude} \label{wave_driver_amplitude}
The maximum voltage the wave driver will use when oscillating, proportional to the displacement of the wave driver arm.

\subsubsection{Sampling Frequency} \label{sampling_frequency}
The time intervals at which the PASCO motion sensor or displacement sensor poll the position or displacement of the beam. Sensor with a higher sampling frequency will collect more data in the same amount of time.

\subsubsection{Smoothing Span} \label{smoothing_span}
The number of elements after a given element used to calculate a rolling mean while data processing. A small smoothing span will better preserve the shape of the data; whereas, a larger smoothing span will better reduce noise and outliers.

\subsubsection{Time} \label{time}
Each test was run for exactly \qty{15}{s}, measured in \qty{0.01}{s} intervals.

\subsection{Dependent Variables} \label{dependent_variables}
\subsubsection{Displacement/Position} \label{displacement-position}
The change in location of the cantilevered beam from its equilibrium position, measured closer to the free end along a perpendicular vertical axis.

\subsubsection{Best Fit Curve} \label{best_fit_curve}
The line of best fit or fit curve is a normalized curve matching the raw or smoothed data and depends on the shape and magnitude of the displacement data.

\section{Work Assignments} \label{work_assignments}
Refer to Table \ref{table:work_assignments} for the distribution of work during this lab.
\begin{table}[ht]
\begin{center}
	\begin{tabular}{| c | c | c | c |}
		\hline
		\multicolumn{1}{| c |}{\textbf{Task}} & \textbf{Matthew} & \textbf{Peter} & \textbf{Natsuki} \\
		\hline
		\multicolumn{4}{| c |}{\textit{Lab Work}} \\
		\hline
		Date Recording & & X & \\
		\hline
		Exp. Setup & X & & \\
		\hline
		Exp. Work & & & X \\
		\hline
		Exp. Clean-Up & X & X & X \\
		\hline
		\multicolumn{4}{| c |}{\textit{Data Processing}} \\
		\hline
		Data Import & X & & \\
		\hline
		Smoothing & X & X & \\
		\hline
		Line of Best Fit & X & X & X\\
		\hline
		\multicolumn{4}{| c |}{\textit{Report}} \\
		\hline
		Introduction & & & X \\
		\hline
		Objectives & & X & \\
		\hline
		Hypothesis & X & & \\
		\hline
		Variables & & X & \\
		\hline
		Materials & & X & \\
		\hline
		Apparatus & & X & \\
		\hline
		Procedures & X & & \\
		\hline
		Data & X & & \\
		\hline
		Analysis & X & X & X \\
		\hline
		Conclusion & & & X \\
		\hline
		References & & & X \\
		\hline
		Appendix & & & X \\
		\hline
		Revisions & X & X & X \\
		\hline
		Editing & X & & \\
		\hline
	\end{tabular}
\end{center}
\caption{Work assignments for lab 01.}
\label{table:work_assignments}
\end{table}

\section{Materials, Apparatus, and Procedures} \label{m_a_p}


\section{Data} \label{data}


\chapter{Conclusion} \label{conclusion-chapter}
\section{Analysis} \label{analysis}


\section{Conclusion} \label{conclusion-section}


\printbibliography[heading=subbibintoc]
\end{document}
